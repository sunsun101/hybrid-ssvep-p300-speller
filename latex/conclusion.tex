\setlength{\footskip}{8mm}

\chapter{CONCLUSION}

The objective of this research was to develop a hybrid speller using a consumer-grade EEG headset, explore its limitations, document the development journey, and study the long-term viability of the ensemble TRCA model. Our findings revealed that replicating the exact work of \cite{xu2020implementing} with the consumer-grade EEG headset we had proved challenging. However, this led us to identify and understand the limitations of the consumer-grade g.tec Unicorn Hybrid Black Headset. Through alternative approaches, we were able to achieve positive results and make progress in developing a functional hybrid speller system. Additionally, we found that the ensemble TRCA model exhibited long-term viability, indicating its potential for practical applications.

The implications of this research are significant for the widespread use of speller systems. Consumer-grade EEG headsets offer a cost-effective and portable alternative to clinical-grade headsets. This research serves as a starting point for future improvements and advancements in the field. With further research and refinement, hybrid spellers using consumer-grade headsets can become more widely accessible to individuals, ultimately benefiting a larger population.

In conclusion, this research successfully tackled the challenge of developing a hybrid speller with a consumer-grade EEG headset. Despite the initial setback in replicating previous work, we identified the limitations of our equipment and devised alternative solutions. Through our findings, we demonstrated the feasibility of developing a hybrid BCI speller using a consumer-grade EEG headset. The long-term viability of the ensemble TRCA model further solidified its potential for practical application. 
